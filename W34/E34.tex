\documentclass{article}
\usepackage{amsmath}
\usepackage{titlesec}
% \usepackage[mathletters]{ucs}
\usepackage{mathtools} %for \abs{x}
\usepackage[warnings-off={mathtools-colon,mathtools-overbracket}]{unicode-math}
% \setmainfont{TeX Gyre Schola}
% \setmathfont{TeX Gyre Schola Math}
\usepackage[utf8x]{inputenc}
\usepackage{fontenc}
\usepackage[margin=1.5in]{geometry}
\usepackage{enumerate}
\newtheorem{theorem}{Theorem}
\usepackage[dvipsnames]{xcolor}
\usepackage{pgfplots}
\pgfplotsset{compat=1.18}
\setlength{\parindent}{0cm}
\usepackage{graphics}
\usepackage{graphicx} % Required for including images
\usepackage{subcaption}
\usepackage{bigintcalc}
\usepackage{pythonhighlight} %for pythonkode \begin{python}   \end{python}
\usepackage{appendix}
\usepackage{arydshln}
\usepackage{physics}
\usepackage{booktabs} 
\usepackage{adjustbox}
\usepackage{mdframed}
\usepackage{relsize}
\usepackage{physics}
\usepackage[thinc]{esdiff}
\usepackage{esint}  %for lukket-linje-integral
\usepackage{xfrac} %for sfrac
\usepackage[colorlinks=true,linktoc=page]{hyperref} %for linker, må ha med hypersetup
\usepackage[noabbrev, nameinlink]{cleveref} % to be loaded after hyperref
% \usepackage{amssymb} %\mathbb{R} for reelle tall, \mathcal{B} for "matte"-font
\usepackage{listings} %for kode/lstlisting
\usepackage{verbatim}
\usepackage{graphicx,wrapfig,lipsum,caption} %for wrapping av bilder
\usepackage[english]{babel}
\usepackage{cancel}
% \usepackage{alphabeta}
\usepackage{mhchem} % for atom notasjon
% \definecolor{codegreen}{rgb}{0,0.6,0}
% \definecolor{codegray}{rgb}{0.5,0.5,0.5}
% \definecolor{codepurple}{rgb}{0.58,0,0.82}
% \definecolor{backcolour}{rgb}{0.95,0.95,0.92}
% \lstdefinestyle{mystyle}{
%     backgroundcolor=\color{backcolour},   
%     commentstyle=\color{codegreen},
%     keywordstyle=\color{magenta},
%     numberstyle=\tiny\color{codegray},
%     stringstyle=\color{codepurple},
%     basicstyle=\ttfamily\footnotesize,
%     breakatwhitespace=false,         
%     breaklines=true,                 
%     captionpos=b,                    
%     keepspaces=true,                 
%     numbers=left,                    
%     numbersep=5pt,                  
%     showspaces=false,                
%     showstringspaces=false,
%     showtabs=false,                  
%     tabsize=2
% }

% \lstset{style=mystyle}
\title{Exercises Week 34}
% \author{Oskar Idland}
\date{}
\begin{document}
\maketitle
% \tableofcontents
\newpage

\section*{Exercise 1}
\subsection*{a)}
\begin{itemize}
    \item An unitary matrix $U$ has the property that $U^{-1} = U^{†} ⇒ U U^{†} = U^{†}U = I$. 
    \item An orthogonal matrix $Q$ has rows and columns which are orthogonal vectors. It has the properties that $Q^{T} = Q^{-1} ⇒ Q Q^{T} = I$. 
\end{itemize}

\subsection*{b)}
A Hermitian matrix has real eigenvalues and real values on its diagonal. It is also symmetric in that $H_{ij} = \overline{H}_{ji} = H_{ji}$

\subsection*{c)}
We begin by our transformation:
\[
\ket{ψ_{p}} = U\ket{ϕ_{λ}}
\]
We check for orthonormality. 
\[
\bra{ψ_{p}}\ket{ψ_{p}} = \bra{ϕ_{λ}}UU^{†}\ket{ϕ_{λ}} = \bra{ϕ_{λ}}\ket{ϕ_{λ}} = 1
\] 



\subsection*{d)}
\[
O\ket{ϕ_{λ}} = o\ket{ϕ_{λ}}
\]
\[
UO\ket{ϕ_{λ}} = oU\ket{ϕ_{λ}}
\]
\[
UO\ket{ϕ_{λ}} = o\ket{ψ_{p}}
\]
\[
UOU^{†}U\ket{ϕ_{λ}} = o\ket{ψ_{p}}
\]
\[
UOU^{†}\ket{ψ_{p}} = o\ket{ψ_{p}}
\]
If $\ket{ϕ_{λ}}$ exist in the Hilbert space $\mathcal{H}_{λ}$ and $\ket{ψ_p}$ in $\mathcal{H}_{p}$, then $U^{†}: \mathcal{H}_p ↦ \mathcal{H}_{λ}$, letting us act with $O: \mathcal{H}_{λ} ↦ \mathcal{H}_{λ}$ on $\ket{ψ_p}$. This is converted back into $\mathcal{H}_{p}$, as $U: \mathcal{H}_{λ} ↦ \mathcal{H}_{p}$. The vectors therefore share eigenvalues. 

\section*{Exercise 2}
\subsection*{a)}





\end{document}

\documentclass[reprint,english,notitlepage]{revtex4-2}
\usepackage{amsmath}
\usepackage[mathletters]{ucs}
\usepackage[utf8x]{inputenc}
\usepackage[english]{babel}
\usepackage{esint}
\usepackage{physics,amssymb}
\usepackage{graphicx}
\usepackage{xcolor}
\usepackage{hyperref}
\usepackage{listings}
\usepackage{subfigure}
% \usepackage[style=science, backend=biber]{biblatex}
% \addbibresource{References_Part_4.bib} TODO: Slett før innlevering


\lstset{inputpath=,
backgroundcolor=\color{white!88!black},
basicstyle={\ttfamily\scriptsize},
commentstyle=\color{magenta},
language=Python,
morekeywords={True,False},
tabsize=4,
stringstyle=\color{green!55!black},
frame=single,
keywordstyle=\color{blue},
showstringspaces=false,
columns=fullflexible,
keepspaces=true}

\begin{document}

\title{}
\author{Oskar Idland }
\date{\today}
\affiliation{Institute of Theoretical Astrophysics, University of Oslo}

\begin{abstract}
    This is an abstract \colorbox{red}{Complete this summary at the end of the paper}
\end{abstract}
\maketitle
\section{Theory} \label{sec: theory}

\section{Method} \label{sec: method}

\section{Results} \label{sec: results}

\section{Discussion} \label{sec: discussion}

\section{Conclusion} \label{sec: conclusion}

\section{Appendix} \label{sec: appendix}

\section*{ACKNOWLEDGMENTS}

\section*{References} \label{sec: references}

\end{document}