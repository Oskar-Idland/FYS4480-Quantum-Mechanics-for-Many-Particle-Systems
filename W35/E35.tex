\documentclass{article}
\usepackage{amsmath}
\usepackage{titlesec}
% \usepackage[mathletters]{ucs}
\usepackage{mathtools} %for \abs{x}
\usepackage[warnings-off={mathtools-colon,mathtools-overbracket}]{unicode-math}
% \setmainfont{TeX Gyre Schola}
% \setmathfont{TeX Gyre Schola Math}
\usepackage[utf8x]{inputenc}
\usepackage{fontenc}
\usepackage[margin=1.5in]{geometry}
\usepackage{enumerate}
\newtheorem{theorem}{Theorem}
\usepackage[dvipsnames]{xcolor}
\usepackage{pgfplots}
\pgfplotsset{compat=1.18}
\setlength{\parindent}{0cm}
\usepackage{graphics}
\usepackage{graphicx} % Required for including images
\usepackage{subcaption}
\usepackage{bigintcalc}
\usepackage{pythonhighlight} %for pythonkode \begin{python}   \end{python}
\usepackage{appendix}
\usepackage{arydshln}
\usepackage{physics}
\usepackage{booktabs} 
\usepackage{adjustbox}
\usepackage{mdframed}
\usepackage{relsize}
\usepackage{physics}
\usepackage[thinc]{esdiff}
\usepackage{esint}  %for lukket-linje-integral
\usepackage{xfrac} %for sfrac
\usepackage[colorlinks=true,linktoc=page]{hyperref} %for linker, må ha med hypersetup
\usepackage[noabbrev, nameinlink]{cleveref} % to be loaded after hyperref
% \usepackage{amssymb} %\mathbb{R} for reelle tall, \mathcal{B} for "matte"-font
\usepackage{listings} %for kode/lstlisting
\usepackage{verbatim}
\usepackage{graphicx,wrapfig,lipsum,caption} %for wrapping av bilder
\usepackage[english]{babel}
\usepackage{cancel}
% \usepackage{alphabeta}
\usepackage{mhchem} % for atom notasjon
% \definecolor{codegreen}{rgb}{0,0.6,0}
% \definecolor{codegray}{rgb}{0.5,0.5,0.5}
% \definecolor{codepurple}{rgb}{0.58,0,0.82}
% \definecolor{backcolour}{rgb}{0.95,0.95,0.92}
% \lstdefinestyle{mystyle}{
%     backgroundcolor=\color{backcolour},   
%     commentstyle=\color{codegreen},
%     keywordstyle=\color{magenta},
%     numberstyle=\tiny\color{codegray},
%     stringstyle=\color{codepurple},
%     basicstyle=\ttfamily\footnotesize,
%     breakatwhitespace=false,         
%     breaklines=true,                 
%     captionpos=b,                    
%     keepspaces=true,                 
%     numbers=left,                    
%     numbersep=5pt,                  
%     showspaces=false,                
%     showstringspaces=false,
%     showtabs=false,                  
%     tabsize=2
% }

% \lstset{style=mystyle}
\author{Oskar Idland}
\title{Exercises Week 35}
\date{}
\begin{document}
\maketitle
\newpage

\section*{Exercise 1}   
\subsection*{a)}
\[
Φ^{AS} = \frac{1}{\sqrt{3!}} ∑_{p=0}^{3!} (-1)^p ∏_{i=1}^{3} ψ_{a_{i}}(x_{i})
\]
\begin{align*}
\frac{1}{\sqrt{6}}\Big( &ψ_{α_{1}}(x_1)ψ_{α_2}(x_2)ψ_{α_3}(x_3) -  ψ_{α_{1}}(x_2)ψ_{α_1}(x_1)ψ_{α_3}(x_3) +  ψ_{α_{1}}(x_2)ψ_{α_2}(x_3)ψ_{α_3}(x_1) \\  -\ &ψ_{α_{1}}(x_3)ψ_{α_2}(x_2)ψ_{α_3}(x_1) +  ψ_{α_{1}}(x_3)ψ_{α_2}(x_1)ψ_{α_3}(x_2) -  ψ_{α_{1}}(x_1)ψ_{α_2}(x_3)ψ_{α_3}(x_2)\Big)
\end{align*}

\subsection*{b)}
\[
\bra{Φ_{λ}^{AS}}\ket{Φ_{λ}^{AS}} = ∫ \left|Φ_{λ}^{AS}(x_1, \ldots  ,x_N, α_1, \ldots  ,α_N)\right|^2 \mathrm{d}τ = 1
\]
We assume the particles-states to be orthonormal. 
\[
\bra{ψ_{α_i}}\ket{ψ_{α_j}} = ∫ ψ^{*}_{α_i}(\mathbf{r}) ψ_{α_j}(\mathbf{r})\mathrm{d}\mathbf{r} = δ_{ij}
\]
Using the antisymmetrizer operator  $\mathcal{A}$ and Hartree-function $ϕ_H$:
\[
\mathcal{A} = \frac{1}{N!} ∑_{p}^{} (-)^{p}\hat{P} \quad , \quad  ϕ_H = ∏_{i=1}^{N} ψ_{α_i}(x_i) 
\]
We can rewrite the Slater determinant to be the following:
\[
Φ_{λ}^{AS} = \sqrt{N!}\mathcal{A}ϕ_H
\]
We now have a new expression for the original inner product:
\[
\bra{Φ_{λ}^{AS}}\ket{Φ_{λ}^{AS}} = N! ∫ \mathcal{A}^{*}ϕ^{*}_H \mathcal{A}ϕ_H \mathrm{d}τ 
\]
Using that $\mathcal{A}^2 = \mathcal{A}$, as $\mathcal{A}$ is Hermitian, and that it is a projection operator we can simplify:
\[
N! ∫ ϕ_H^{*} \mathcal{A}ϕ_H \mathrm{d}τ = ∑_{p}^{} (-)^{p} ∫ ϕ_H^{*} \hat{P} ϕ_H \mathrm{d}τ 
\]
The permutation operator acts on the Hartree-function, which makes all it's states different. The only contribution comes when the permutation is identity. 
\[
\bra{Φ_{λ}^{AS}}\ket{Φ_{λ}^{AS}} = ∫ ϕ^{*}_H ϕ_H \mathrm{d}τ = 1
\]

\subsection*{c)}
\paragraph{Onebody Operator:}
\[
\bra{Φ_{α_1 α_2}^{AS}} \hat{F} \ket{Φ_{α_1 α_2}^{AS}} = 2 ∫ ψ^{*}_{α_1}(x_1)ψ^{*}_{α_2}(x_2) \hat{F} \mathcal{A} ψ_{α_1}(x_1) ψ_{α_2}(x_2) \mathrm{d}τ
\]
Inserting its the operators definition:
\[
\bra{Φ_{α_1 α_2}^{AS}} \hat{F} \ket{Φ_{α_1 α_2}^{AS}} = ∑_{p}^{} (-)^{p} ∫ ψ^{*}_{α_1}(x_1)ψ^{*}_{α_2}(x_2) \hat{f}(x_1) \hat{P} ψ_{α_1}(x_1)ψ_{α_2}(x_2) \mathrm{d}τ + ∑_{p}^{} (-)^{p} ∫ ψ^{*}_{α_1}(x_1)ψ^{*}_{α_2}(x_2) \hat{f}(x_2) \hat{P} ψ_{α_1}(x_1)ψ_{α_2}(x_2) \mathrm{d}τ     
\]
The Hartree-function, when permuted, will make the integral vanish. There is only one term of each sum left. 
\[
\bra{Φ_{α_1 α_2}^{AS}} \hat{F} \ket{Φ_{α_1 α_2}^{AS}} = ∫ ψ^{*}_{α_1}(x_1)ψ^{*}_{α_2}(x_2) \hat{f}(x_1)  ψ_{α_1}(x_1)ψ_{α_2}(x_2) \mathrm{d}τ + ∫ ψ^{*}_{α_1}(x_1)ψ^{*}_{α_2}(x_2) \hat{f}(x_2)  ψ_{α_1}(x_1)ψ_{α_2}(x_2) \mathrm{d}τ   
\]
As the onebody operator only acts on a single particle, we can integrate over the other coordinate. The single-particle states are orthonormal, giving a unitary integral. 
\[
\bra{Φ_{α_1 α_2}^{AS}} \hat{F} \ket{Φ_{α_1 α_2}^{AS}} = ∫ ψ^{*}_{α_1}(\mathbf{r}) \hat{f}(\mathbf{r})ψ_{α_1}(\mathbf{r}) \mathrm{d}\mathbf{r} +  ∫ ψ^{*}_{α_2}(\mathbf{r}) \hat{f}(\mathbf{r})ψ_{α_2}(\mathbf{r}) \mathrm{d}\mathbf{r} 
\]
With this we get a general expression for a onebody operator:
\[
\bra{α_i}\hat{q}\ket{α_j} ≡  ∫ ψ^{*}_{α_i}(\mathbf{r}) \hat{q}(\mathbf{r})ψ_{α_j}(\mathbf{r}) \mathrm{d}\mathbf{r}
\]
This gives:
\[
\bra{Φ_{α_1 α_2}^{AS}}\hat{f}\ket{Φ_{α_1 α_2}^{AS}} ≡ \bra{α_1}\hat{f}\ket{α_1} + \bra{α_2}\hat{f}\ket{α_2}
\]

\paragraph{Twobody Operator:}
\[
\bra{Φ_{α_1 α_2}^{AS}}\hat{G}\ket{Φ_{α_1 α_2}^{AS}} = 2∫ ψ^{*}_{α_1}(x_1) ψ^{*}_{α_2}(x_2) \hat{G} \mathcal{A}ψ_{α_1}(x_2)ψ_{α_2}(x_2) \mathrm{d}τ
\]
Given that there generally only are two particles, we insert the definition of $\hat{G}$. 
\[
\bra{Φ_{α_1 α_2}^{AS}} \hat{G} \ket{Φ_{α_1 α_2}^{AS}} 2 ∫ ψ^{*}_{α_1}(x_1)ψ^{*}_{α_2}(x_2) \hat{G} \mathcal{A} ψ_{α_1}(x_1)ψ_{α_2}(x_2)\mathrm{d}τ
\]
\[
\bra{Φ_{α_1 α_2}^{AS}} \hat{G} \ket{Φ_{α_1 α_2}^{AS}} = ∑_{p}^{} (-)^{p} ∫  ψ^{*}_{α_1}(x_1)ψ^{*}_{α_2}(x_2)\hat{P} ψ_{α_1}(x_1)ψ_{α_2}(x_2)\mathrm{d}τ
\]
Writing out the permutations:
\begin{align*}
\bra{Φ_{α_1 α_2}^{AS}} \hat{G} \ket{Φ_{α_1 α_2}^{AS}} = &∫ ψ^{*}_{α_1}(x_1)ψ^{*}_{α_2}(x_2)\hat{g}(x_1, x_2)ψ_{α_1}(x_1)ψ_{α_2}(x_2)\mathrm{d}τ  \\ -&∫ ψ^{*}_{α_1}(x_1)ψ^{*}_{α_2}(x_2)\hat{g}(x_1, x_2)\underbrace{ψ_{α_1}(x_2)ψ_{α_2}(x_1)}_{\text{switched pos.}}\mathrm{d}τ
\end{align*}
The first term is the Hartree-term, the second is the Fock-term. We can use a shorthand notation:
\[
\bra{αβ} \hat{q} \ket{γδ} ≡ ∫  ψ^{*}_{α_1}(x_1)ψ^{*}_{β}(x_2) \hat{q}(x_1, x_2) ψ_{γ}(x_1)ψ_{δ}(x_2)\mathrm{d}x_1 \mathrm{d}x_2
\]
With:
\[
\bra{αβ} \hat{q} \ket{γδ}_{AS} = \bra{αβ} \hat{q} \ket{γδ} + \bra{αβ} \hat{q} \ket{δγ} 
\]
This finally gives:
\[
\bra{Φ_{α_1 α_2}^{AS}} \hat{G} \ket{Φ_{α_1 α_2}^{AS}} = \bra{α_1 α_2} \hat{g} \ket{α_1 α_2}_{AS}
\]
As the particles are indistinguishable, the operators must have permutation symmetry, making them commute with the permutation operator. They should also be Hermitian, as they correspond to physical observables. 
\[
\left[\mathcal{A}, \hat{F}\right] = \left[\mathcal{A}, \hat{G}\right] = 0
\]

\section*{Exercise 2}
\subsection*{a)}
There are three states. One particle with spin up and one with spin down. As the particles are indistinguishable, we do not care which particle is which. We therefore have a total of six slater determinants: 
\[
11 \ 12 \ 13 \ 23 \ 33 \ 22
\]
If the particles must be in the same state, we have only three slater determinants
\[
11 \ 22 \ 33
\]

\subsection*{b)}
Each element of the matrix representation of the Hamiltonian in the basis of $\left\{Φ_0, Φ_1\right\}$ is defined as following:
\[
\begin{pmatrix*}[r]
 \bra{Φ_0} \hat{H} \ket{Φ_0} &  \bra{Φ_0} \hat{H} \ket{Φ_1} \\
 \bra{Φ_1} \hat{H} \ket{Φ_0} &  \bra{Φ_1} \hat{H} \ket{Φ_1} \\
\end{pmatrix*}
\]
The off-diagonal elements have energie values from the particle interactions, which we know have a value of $-g$. Using the definition of $\hat{H}_0$, we clearly see that the energie from this part of the total Hamiltonian, is $2d$ and $4d$. 
\[
\hat{H} = 
\begin{pmatrix*}[r]
 2d-g & -g \\
 -g & 4d-g \\
\end{pmatrix*}
\]

To find the eigenvalues we solve the following:
\[
\left(\hat{H} - λI\right) \ket{Ψ} = 0 ⇒ \det\left(\hat{H} - λI\right) = 0
\]
\[
λ^2 + (2g - 6d)λ + 8d^2 - 24dg
\]
\[
λ_{±} = 3d - g ± \sqrt{g^2 + d^2}
\]

Adding it back in to the equation and diagonalizing we get:
\[
\begin{pmatrix*}[r]
 \frac{d ± \sqrt{d^2 + g^2}}{g} & 0 \\
 0 & 1 \\
\end{pmatrix*}
\]
Now we define our eigenstates from our basis:
\[
\ket{Ψ_0} = \frac{d + \sqrt{d^2 + g^2}}{g} \ket{Φ_0} \ket{Φ_1}
\]
\[
\ket{Ψ_1} = \frac{d - \sqrt{d^2 + g^2}}{g} \ket{Φ_0} \ket{Φ_1}
\]
For simplification we add the new variable $γ ≡ d/g$. It represents the ratio between the energy from the interaction and the energy level. Using this on the eigenstates:
\[
\ket{Ψ_0} = \left(γ + \sqrt{1 + γ^2}\right)\ket{Φ_0} + \ket{Φ}
\]
\[
\ket{Ψ_1} = \left(γ - \sqrt{1 + γ^2}\right)\ket{Φ_0} + \ket{Φ_1}
\]

\subsection*{c)}
With $p = 3$, we only get a new element on the diagonal for the Hamiltonian
\[
\begin{pmatrix*}[r]
 2d-g & -g & -g \\
 -g & 4d-g & -g \\
 -g & -g & 6d-g \\
\end{pmatrix*} = 
\begin{pmatrix*}[r]
 2γ-1 & -1 & -1 \\
 -1 & 4γ-1 & -1 \\
 -1 & -1 & 6γ-1 \\
\end{pmatrix*}
\]


\end{document}